\maketitle

\section{Introduction to 3D Rendering Engines}
3D rendering engines are powerful tools for generating photorealistic or stylized visuals. These engines work by transforming 3D models and scenes, composed of geometry, materials, lighting, and cameras, into 2D images. The process involves three key stages:

\begin{enumerate}
    \item \textbf{Modeling}: Defining the geometry of objects.
    \item \textbf{Material and Lighting}: Applying surface textures and simulating light interactions.
    \item \textbf{Rendering}: Converting 3D information into a 2D projection using algorithms like ray tracing or rasterization.
\end{enumerate}

\section{Adapting Rendering Principles for SVG Illustration}
SVG (Scalable Vector Graphics) engines deal with 2D vector graphics. While simpler than 3D rendering, principles from 3D rendering can enhance an SVG illustration engine by:

\begin{itemize}
    \item Employing geometric set operations to create complex shapes.
    \item Simulating lighting and shadows using gradients or blending techniques.
    \item Layering and compositing elements to achieve depth and complexity.
\end{itemize}

Python libraries like \texttt{shapely} are particularly suited for such tasks due to their robust geometric manipulation capabilities.

\section{Using Python and Shapely for Geometric Set Operations}
\texttt{Shapely} is a Python library designed for manipulation and analysis of planar geometric objects. It supports operations such as union, intersection, and difference, which are fundamental for constructing SVG illustrations.

Here is a simple example of using \texttt{Shapely} to create a composite shape:

\inputminted{python}{rendering/shading/masking.py}

\textbf{Explanation:} This script demonstrates how to combine geometric primitives (e.g., rectangles, circles) using union operations to create a more complex shape.

\section{A Python-Based SVG Illustration Engine}
Below is a high-level overview of how a Python-based SVG engine could function:

\begin{enumerate}
    \item \textbf{Input Parsing:} Accept SVG commands or geometric primitives as input.
    \item \textbf{Geometry Processing:} Use \texttt{Shapely} to perform operations like union, intersection, and difference.
    \item \textbf{Rendering to SVG:} Convert processed geometry into SVG elements and attributes.
\end{enumerate}

Here is an example Python function that integrates \texttt{Shapely} with SVG rendering:

\begin{minted}{python}
from shapely.geometry import Point, Polygon
from shapely.ops import unary_union

def create_svg():
    circle = Point(0, 0).buffer(1)  # Create a circle
    square = Polygon([(-1, -1), (1, -1), (1, 1), (-1, 1)])  # Create a square

    combined_shape = unary_union([circle, square])  # Perform a union operation

    # Convert to SVG path (mock function for demonstration)
    svg_path = combined_shape.svg_path()
    return f'<svg>{svg_path}</svg>'

print(create_svg())
\end{minted}

\textbf{Explanation:} The function combines a circle and a square using a union operation and outputs an SVG string.

\section{Conclusion}
By leveraging principles from 3D rendering and libraries like \texttt{Shapely}, we can create powerful and flexible SVG illustration engines. These engines can perform complex geometric operations, enabling the generation of intricate vector graphics for a variety of applications.
