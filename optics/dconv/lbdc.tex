\documentclass{article}
\usepackage[T1]{fontenc}

\begin{document}

\section*{The Lebesgue Dominated Convergence Theorem: A High-Level Overview}
The Lebesgue Dominated Convergence Theorem (LDCT) is a cornerstone of measure theory and integral calculus. It provides conditions under which the limit of an integral can be exchanged with the integral of a limit. This is particularly useful in mathematical analysis, probability theory, and applied fields like physics and engineering.

In essence, LDCT states that if a sequence of functions \(f_n\) converges pointwise to a function \(f\), and there exists a dominating function \(g\) such that \(|f_n(x)| \leq g(x)\) for all \(n\) and \(x\), and \(g\) is integrable (i.e., \(\int g < \infty\)), then:
\[
\lim_{n \to \infty} \int f_n(x) \, dx = \int \lim_{n \to \infty} f_n(x) \, dx.
\]

This theorem elegantly combines the concepts of convergence, domination, and integration, ensuring the transition from pointwise to integral limits is valid. Its applications range from simplifying complex integrals to proving convergence results in stochastic processes and partial differential equations.

\section*{Conclusion}
This project marries scientific principles with artistic creativity, exploring how light, color, and geometry shape our perception and expression. By examining the connections between halftones, color spaces, and computational techniques, we gain deeper appreciation for the shared language of art and science, rooted in the universal interplay of light and shadow.

\end{document}
