\chapter{Optics}
The study of light and its interaction with matter has been foundational to both science and art. Newton's groundbreaking work, often referred to as "Opticks," explored the behavior of light and laid the groundwork for modern optics.

Key topics include:
\begin{itemize}
    \item \textbf{Reflection and Refraction}: How light interacts with surfaces.
    \item \textbf{Lens Design}: Applications in nanolithography and imaging.
    \item \textbf{Wave Interference}: Artistic use of diffraction patterns.
\end{itemize}
See the detailed write-up in \href{./optics.pdf}{optics.pdf}.

\section{Halftones and Printing}
Halftones create the illusion of continuous tone imagery through dots of varying size and spacing. This technique bridges the analog and digital worlds. For example:
\begin{itemize}
    \item \textbf{CMYK Printing}: Subtractive color mixing.
    \item \textbf{RGB Displays}: Additive color mixing.
    \item \textbf{Moire Patterns}: Undesired interference or creative effects.
\end{itemize}
See illustrations in the \href{./figures}{figures directory}.

\section{Generative Art and Machine Learning}
Generative art leverages algorithms to create intricate patterns and images. Key methods include:
\begin{itemize}
    \item \textbf{Sobel Filters}: Used for edge detection in images.
    \item \textbf{Stochastic Gradient Descent (SGD)}: Optimizing blank grids to match gradients.
    \item \textbf{Recursive Algorithms}: Generating fractal-like designs.
\end{itemize}
For code and examples, see the \href{./notebooks}{notebooks directory}.

\section{Resources and References}
Here are links to additional resources and files:
\begin{itemize}
    \item \href{./workingish\ latex}{Working LaTeX files}
    \item \href{./svgs}{SVG illustrations}
    \item \href{./dconv}{Related conversions}
\end{itemize}

\section{Conclusion}
The synergy between science and art enriches our understanding of both fields. By examining their intersections, we uncover new ways to innovate and inspire.

\end{document}
