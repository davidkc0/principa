\documentclass{article}
\usepackage{amsmath, amssymb, graphicx, hyperref}
\usepackage{geometry}
\geometry{a4paper, margin=1in}
\title{An Exploration of Light, Optics, Halftones, and Generative Art}
\author{}
\date{}

\begin{document}
\maketitle

\section*{Introduction}
Light and its interaction with matter form the foundation of both artistic expression and scientific inquiry. From the intricate physics of optics to the computational techniques of halftones and image replication, the interplay of light, color, and geometry reveals profound insights into our understanding of the visual world. This essay introduces the central themes of this project, connecting the scientific and artistic domains to explore how ideas like CMYK versus RGB, moir\'{e} patterns, and Sobel-filtered gradient matching are utilized in generative art.

\section*{The Science of Light and Optics}
Light, as both wave and particle, has captivated scientists for centuries. Optics, the study of light's behavior, provides tools to understand phenomena such as reflection, refraction, and diffraction. These principles are pivotal in modern technology, from lasers to lenses. The ability to manipulate light also underpins artistic techniques, allowing the creation of depth, contrast, and texture in visual representations.

\subsection*{Color Spaces: RGB and CMYK}
Color is a critical aspect of visual art and design, and its representation involves distinct mathematical frameworks. RGB (Red, Green, Blue) and CMYK (Cyan, Magenta, Yellow, Key/Black) are two primary color spaces used in digital screens and print, respectively. RGB combines emitted light to create colors, relying on additive color mixing. CMYK, in contrast, subtracts light using inks or pigments, making it better suited for physical media. Understanding these systems is vital for converting designs between digital and print formats without loss of fidelity.

\section*{Halftones and Moire Patterns}
Halftones approximate continuous tones in printed images by using dots of varying sizes and densities. This method exploits the human eye's tendency to blend small details into a coherent whole. However, when overlapping halftone patterns occur, they may produce moir\'{e} patterns: interference effects that result in unexpected visual artifacts. While often undesirable, these patterns can also inspire creative exploration in generative art.

\section*{Generative Art and Gradient Matching}
Generative art bridges the gap between technology and creativity, using algorithms to produce designs. One innovative approach involves Sobel filters, which detect image gradients by emphasizing edges. By employing stochastic gradient descent (SGD), a blank grid can deform to match these gradients, effectively replicating an image's structure. This technique highlights the intersection of mathematics, computation, and aesthetics, enabling artists to push boundaries and reimagine traditional concepts.

\section*{Integrating Science and Art}
The project's files showcase various explorations into these themes:

\begin{itemize}
    \item The Jupyter notebook delves into procedural designs, employing libraries such as Shapely and NumPy to generate geometrical patterns.
    \item The SVG file represents stippled designs, demonstrating the application of computational geometry in art.
    \item Photographic examples illustrate how light and optics influence real-world visuals, setting the stage for generative reinterpretations.
\end{itemize}

\section*{Conclusion}
This project marries scientific principles with artistic creativity, exploring how light, color, and geometry shape our perception and expression. By examining the connections between halftones, color spaces, and computational techniques, we gain deeper appreciation for the shared language of art and science, rooted in the universal interplay of light and shadow.

\end{document}
